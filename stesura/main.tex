\documentclass[a4paper,12pt,dvips]{thesis}
\usepackage[latin1]{inputenc}
\usepackage[italian]{babel}
\usepackage{amsfonts}
\usepackage{amssymb,amsmath}
\usepackage[pdftex]{graphicx}
\usepackage{latexsym}

\newtheorem{theorem}{Teorema}[section]
\newtheorem{lemma}{Lemma}[section]
\newtheorem{corollary}{Corollario}[section]
\newtheorem{remark}{Osservazione}[section]
\newtheorem{definition}{Definizione}[section]


%%%%%%%%%%%%%%%%%%%%%%%%%%%%%%%%%%%%%%%%%%%%%%%%%%%%%%%%%%%%%%%%%%%%%%%%%%%%%%%%%%%%%%%%%
% Macro algoritmi in un frame, con caption e label. Si usano cosi':



\newcommand{\algoritmo}{
\begin{figure}[htbp]%
\begin{center}%
\begin{framepage}{\textwidth}%
\singlespace\small%
\begin{tabbing}%
}

\newcommand{\finealgoritmo}[2]{
\end{tabbing}%
\end{framepage}%
\end{center}%
\caption{#2}%
\label{fi:#1}%
\end{figure}%
\noindent%
}

\newenvironment{chapterAbstract}
         {\singlespace \begin{quote}\begin{itshape}\begin{small}  \hugeinitial}
         {\newline \line(1,0){200} \end{small}\end{itshape}\end{quote} \onehalfspace}
                                           

\university{Firenze} \faculty{Ingegneria} \dept{Ingegneria dell'Informazione} \course{Ingegneria Informatica}
\accademicyear{2019 - 2020} 
\supervisor{Marco Bertini}
\supervisor{Secondo Supervisore}
\advisor{Correlatore 1}
\author{Andrea Moscatelli}
\title{Riconoscimento di movimenti corporei tramite concatenazioni di reti neurali}

%%%%%%%%%%%%%%%%%%%%%%%%%%%%%%%%%%%%%%%%%%%%%%%%%%%%%%%%%%%%%%%%%%%%%%%%%%%%%%%%%%%%%%%%%
\def\conclusionname{Conclusioni}
\def\conclusion{
  \chapter*{\conclusionname
        %\@mkboth{\uppercase{\conclusionname}}{\uppercase{\conclusionname}}
        }%
  \addcontentsline{toc}{chapter}{\conclusionname}%
}



\begin{document}
\sffamily
\maketitle

\onehalfspace
\oddsidemargin  1.75cm 
\evensidemargin 1.75cm
\hyphenation{words}

\tableofcontents

\newpage

\algoritmo
1. \= iscriviti ad Ingegneria \\
2. \> finch\a'e non ti stufi o non finisci gli esami \\
   \> 2.1. \= prova a dare l' esame $i$ \\
   \> 2.2. \> se superi l'esame $i$ \\
   \>      \> 2.2.1. \= $i=i+1$ \\
3. \> prepara la tesi oppure fattela fare (CEPU) \\
4. \> scrivila in \LaTeX :) \\
5. \> laureati
\finealgoritmo{laurea}{Algoritmo per conseguire la Laurea in Ingegneria}


 ora si puo' fare riferimento all' algoritmo con \ref{fi:laurea}.





%%%%%%%%%%%%%%%%%%%%%%%%%%%%%%%%%%%%%%%%%%%%%%%%%%%%%%%%%%%%%%%%%%%%%%%%%%%%%%%%%%%%%%%%%



\preface{
prefazione
}

\introduction{
Introduzione
}

\chapter{DensePose}
\section{COCO-Dataset}
\section{Struttura della rete}

\chapter{Detectron2}
Descrizione di cos� Detectron2

\chapter{PoseNet}
\section{Struttura della rete}

\chapter{Classificazione}
\section{Struttura della rete}
\section{Tecniche}

\subsection{Semplice}
\subsection{Tecnica dei centri}
\subsection{Tecnica delle differenze}

\chapter{Risultati ottenuti}

\chapter{Conclusioni}

\chapter{Sviluppi futuri}

\thebibliography{}

\end{document}